% generated by GAPDoc2LaTeX from XML source (Frank Luebeck)
\documentclass[a4paper,11pt]{report}

\usepackage[top=37mm,bottom=37mm,left=27mm,right=27mm]{geometry}
\sloppy
\pagestyle{myheadings}
\usepackage{amssymb}
\usepackage[utf8]{inputenc}
\usepackage{makeidx}
\makeindex
\usepackage{color}
\definecolor{FireBrick}{rgb}{0.5812,0.0074,0.0083}
\definecolor{RoyalBlue}{rgb}{0.0236,0.0894,0.6179}
\definecolor{RoyalGreen}{rgb}{0.0236,0.6179,0.0894}
\definecolor{RoyalRed}{rgb}{0.6179,0.0236,0.0894}
\definecolor{LightBlue}{rgb}{0.8544,0.9511,1.0000}
\definecolor{Black}{rgb}{0.0,0.0,0.0}

\definecolor{linkColor}{rgb}{0.0,0.0,0.554}
\definecolor{citeColor}{rgb}{0.0,0.0,0.554}
\definecolor{fileColor}{rgb}{0.0,0.0,0.554}
\definecolor{urlColor}{rgb}{0.0,0.0,0.554}
\definecolor{promptColor}{rgb}{0.0,0.0,0.589}
\definecolor{brkpromptColor}{rgb}{0.589,0.0,0.0}
\definecolor{gapinputColor}{rgb}{0.589,0.0,0.0}
\definecolor{gapoutputColor}{rgb}{0.0,0.0,0.0}

%%  for a long time these were red and blue by default,
%%  now black, but keep variables to overwrite
\definecolor{FuncColor}{rgb}{0.0,0.0,0.0}
%% strange name because of pdflatex bug:
\definecolor{Chapter }{rgb}{0.0,0.0,0.0}
\definecolor{DarkOlive}{rgb}{0.1047,0.2412,0.0064}


\usepackage{fancyvrb}

\usepackage{mathptmx,helvet}
\usepackage[T1]{fontenc}
\usepackage{textcomp}


\usepackage[
            pdftex=true,
            bookmarks=true,        
            a4paper=true,
            pdftitle={Written with GAPDoc},
            pdfcreator={LaTeX with hyperref package / GAPDoc},
            colorlinks=true,
            backref=page,
            breaklinks=true,
            linkcolor=linkColor,
            citecolor=citeColor,
            filecolor=fileColor,
            urlcolor=urlColor,
            pdfpagemode={UseNone}, 
           ]{hyperref}

\newcommand{\maintitlesize}{\fontsize{50}{55}\selectfont}

% write page numbers to a .pnr log file for online help
\newwrite\pagenrlog
\immediate\openout\pagenrlog =\jobname.pnr
\immediate\write\pagenrlog{PAGENRS := [}
\newcommand{\logpage}[1]{\protect\write\pagenrlog{#1, \thepage,}}
%% were never documented, give conflicts with some additional packages

\newcommand{\GAP}{\textsf{GAP}}

%% nicer description environments, allows long labels
\usepackage{enumitem}
\setdescription{style=nextline}

%% depth of toc
\setcounter{tocdepth}{1}

\usepackage{amsmath}



%% command for ColorPrompt style examples
\newcommand{\gapprompt}[1]{\color{promptColor}{\bfseries #1}}
\newcommand{\gapbrkprompt}[1]{\color{brkpromptColor}{\bfseries #1}}
\newcommand{\gapinput}[1]{\color{gapinputColor}{#1}}


\begin{document}

\logpage{[ 0, 0, 0 ]}
\begin{titlepage}
\mbox{}\vfill

\begin{center}{\maintitlesize \textbf{ TwistedConjugacy \mbox{}}}\\
\vfill

\hypersetup{pdftitle= TwistedConjugacy }
\markright{\scriptsize \mbox{}\hfill  TwistedConjugacy  \hfill\mbox{}}
{\Huge \textbf{ Computation with twisted conjugacy classes \mbox{}}}\\
\vfill

{\Huge  2.4.0 \mbox{}}\\[1cm]
{ 23 November 2024 \mbox{}}\\[1cm]
\mbox{}\\[2cm]
{\Large \textbf{ Sam Tertooy\\
    \mbox{}}}\\
\hypersetup{pdfauthor= Sam Tertooy\\
    }
\end{center}\vfill

\mbox{}\\
{\mbox{}\\
\small \noindent \textbf{ Sam Tertooy\\
    }  Email: \href{mailto://sam.tertooy@kuleuven.be} {\texttt{sam.tertooy@kuleuven.be}}\\
  Homepage: \href{https://stertooy.github.io/} {\texttt{https://stertooy.github.io/}}\\
  Address: \begin{minipage}[t]{8cm}\noindent
 Wiskunde\\
 KU Leuven Kulak Kortrijk Campus\\
 Etienne Sabbelaan 53\\
 8500 Kortrijk\\
 Belgium\\
 \\
 \end{minipage}
}\\
\end{titlepage}

\newpage\setcounter{page}{2}
{\small 
\section*{Abstract}
\logpage{[ 0, 0, 1 ]}
 The \textsc{TwistedConjugacy} package provides methods to calculate Reidemeister classes, numbers, spectra
and zeta functions, as well as other methods related to homomorphisms,
endomorphisms and automorphisms of groups. These methods are, for the most
part, designed to be used with finite groups and polycyclically presented
groups. \mbox{}}\\[1cm]
{\small 
\section*{Copyright}
\logpage{[ 0, 0, 2 ]}
 {\copyright} 2020\texttt{\symbol{45}}2024 Sam Tertooy 

 The \textsc{TwistedConjugacy} package is free software, it may be redistributed and/or modified under the
terms and conditions of the \href{ https://www.gnu.org/licenses/old-licenses/gpl-2.0.en.html} {GNU Public License Version 2} or (at your option) any later version. \mbox{}}\\[1cm]
{\small 
\section*{Acknowledgements}
\logpage{[ 0, 0, 3 ]}
 This documentation was created using the \textsc{GAPDoc} and \textsc{AutoDoc} packages. \mbox{}}\\[1cm]
\newpage

\def\contentsname{Contents\logpage{[ 0, 0, 4 ]}}

\tableofcontents
\newpage

     
\chapter{\textcolor{Chapter }{Preface}}\label{Chapter_preface}
\logpage{[ 1, 0, 0 ]}
\hyperdef{L}{X874E1D45845007FE}{}
{
  Let $G, H$ be groups and $\varphi,\psi\colon H \to G$ group homomorphisms. Then the pair $(\varphi,\psi)$ induces a (right) group action on $G$ given by 
\[G \times H \to G\colon (g,h) \mapsto g \cdot h = \psi(h)^{-1} g\varphi(h).\]
 This group action is called \emph{$(\varphi,\psi)$\texttt{\symbol{45}}twisted conjugation}, and induces an equivalence relation $\sim_{\varphi,\psi}$ on $G$: 
\[g_1 \sim_{\varphi,\psi} g_2 \iff \exists h \in H: g_1 \cdot h = g2.\]
 The equivalence classes (i.e. the orbits of the action) are called \emph{Reidemeister classes} and the number of Reidemeister classes is called the \emph{Reidemeister number} $R(\varphi,\psi)$ of the pair $(\varphi,\psi)$. The stabiliser of the identity $1_G$ for this action is the \emph{coincidence group} $\operatorname{Coin}(\varphi, \psi )$, i.e. the subgroup of $H$ given by 
\[\operatorname{Coin}(\varphi,\psi) := \{\, h \in H \mid \varphi(h) = \psi(h)
\,\}.\]
 

 The \textsc{TwistedConjugacy} package provides methods to calculate Reidemeister classes, Reidemeister
numbers and coincidence groups of pairs of group homomorphisms. These methods
are implemented for finite groups and polycyclically presented groups. If $H$ and $G$ are both infinite polycyclically presented groups, then some methods in this
package are only guaranteed to produce a result if either $G = H$ or $G$ is nilpotent\texttt{\symbol{45}}by\texttt{\symbol{45}}finite. Otherwise, these
methods may potentially throw an error: "\texttt{Error, no method found!}" 

 Bugs in this package, in \textsc{GAP} or any other package used directly or indirectly, may cause functions from
this package to produce errors or even wrong results. You can set the variable \texttt{ASSERT@TwistedConjugacy} to \texttt{true}, which will cause certain functions to verify the correctness of their
output. This should make results more (but not completely!) reliable, at the
cost of some performance. 

 When using this package with PcpGroups, you can do the same for \textsc{Polycyclic}'s variables \texttt{CHECK{\textunderscore}CENT@Polycyclic}, \texttt{CHECK{\textunderscore}IGS@Polycyclic} and \texttt{CHECK{\textunderscore}INTSTAB@Polycyclic}. }

   
\chapter{\textcolor{Chapter }{Twisted Conjugacy}}\label{Chapter_twicon}
\logpage{[ 2, 0, 0 ]}
\hyperdef{L}{X78DFA75A82655B7F}{}
{
  
\section{\textcolor{Chapter }{Twisted Conjugation Action}}\label{Chapter_Twisted_Conjugacy_Section_Twisted_Conjugation_Action}
\logpage{[ 2, 1, 0 ]}
\hyperdef{L}{X7BD5FC777FC1B7A4}{}
{
  Let $G, H$ be groups and $\varphi,\psi\colon H \to G$ group homomorphisms. Then the pair $(\varphi,\psi)$ induces a (right) group action on $G$ given by 
\[G \times H \to G\colon (g,h) \mapsto g \cdot h := \psi(h)^{-1} g\varphi(h).\]
 This group action is called \emph{$(\varphi,\psi)$\texttt{\symbol{45}}twisted conjugation}, and induces an equivalence relation on the group $G$. We say that $g_1, g_2 \in G$ are $(\varphi,\psi)$\texttt{\symbol{45}}twisted conjugate, denoted by $g_1 \sim_{\varphi,\psi} g_2$, if and only if there exists some element $h \in H$ such that $g_1 \cdot h = g_2$, or equivalently $g_1 = \psi(h) g_2 \varphi(h)^{-1}$. 

If $\varphi\colon G \to G$ is an endomorphism of a group $G$, then by \emph{$\varphi$\texttt{\symbol{45}}twisted conjugacy} we mean $(\varphi,\operatorname{id}_G)$\texttt{\symbol{45}}twisted conjugacy. Most functions in this package will
allow you to input a single endomorphism instead of a pair of homomorphisms.
The "missing" endomorphism will automatically be assumed to be the identity
mapping. Similarly, if a single group element is given instead of two, the
second will be assumed to be the identity. 

\subsection{\textcolor{Chapter }{TwistedConjugation}}
\logpage{[ 2, 1, 1 ]}\nobreak
\label{TwistedConjugationGroup}
\hyperdef{L}{X79CF6BDA7851496D}{}
{\noindent\textcolor{FuncColor}{$\triangleright$\enspace\texttt{TwistedConjugation({\mdseries\slshape hom1[, hom2]})\index{TwistedConjugation@\texttt{TwistedConjugation}}
\label{TwistedConjugation}
}\hfill{\scriptsize (function)}}\\


 Implements the twisted conjugation (right) group action induced by the pair of
homomorphisms ( \mbox{\texttt{\mdseries\slshape hom1}}, \mbox{\texttt{\mdseries\slshape hom2}} ) as a function. }

 

\subsection{\textcolor{Chapter }{RepresentativeTwistedConjugation}}
\logpage{[ 2, 1, 2 ]}\nobreak
\label{IsTwistedConjugateGroup}
\hyperdef{L}{X8493E3818276A562}{}
{\noindent\textcolor{FuncColor}{$\triangleright$\enspace\texttt{RepresentativeTwistedConjugation({\mdseries\slshape hom1[, hom2], g1[, g2]})\index{RepresentativeTwistedConjugation@\texttt{RepresentativeTwistedConjugation}}
\label{RepresentativeTwistedConjugation}
}\hfill{\scriptsize (function)}}\\


 Tests whether the elements \mbox{\texttt{\mdseries\slshape g1}} and \mbox{\texttt{\mdseries\slshape g2}} are twisted conjugate under the twisted conjugacy action of the pair of
homomorphisms ( \mbox{\texttt{\mdseries\slshape hom1}}, \mbox{\texttt{\mdseries\slshape hom2}} ). 

 This function relies on the output of \texttt{RepresentativeTwistedConjugation}. Computes an element that maps \mbox{\texttt{\mdseries\slshape g1}} to \mbox{\texttt{\mdseries\slshape g2}} under the twisted conjugacy action of the pair of homomorphisms ( \mbox{\texttt{\mdseries\slshape hom1}}, \mbox{\texttt{\mdseries\slshape hom2}} ) or returns \texttt{fail} if no such element exists. 

 If $G$ is abelian, this function relies on (a generalisation of) \cite[Alg. 4]{dt21-a}. If $H$ is finite, it relies on a stabiliser\texttt{\symbol{45}}orbit algorithm.
Otherwise, it relies on a mixture of the algorithms described in \cite[Thm. 3]{roma16-a}, \cite[Sec. 5.4]{bkl20-a}, \cite[Sec. 7]{roma21-a} and \cite[Alg. 6]{dt21-a}. }

 
\begin{Verbatim}[commandchars=!@|,fontsize=\small,frame=single,label=Example]
  !gapprompt@gap>| !gapinput@G := AlternatingGroup( 6 );;|
  !gapprompt@gap>| !gapinput@H := SymmetricGroup( 5 );;|
  !gapprompt@gap>| !gapinput@phi := GroupHomomorphismByImages( H, G, [ (1,2)(3,5,4), (2,3)(4,5) ],|
  !gapprompt@>| !gapinput@ [ (1,2)(3,4), () ] );;|
  !gapprompt@gap>| !gapinput@psi := GroupHomomorphismByImages( H, G, [ (1,2)(3,5,4), (2,3)(4,5) ],|
  !gapprompt@>| !gapinput@ [ (1,4)(3,6), () ] );;|
  !gapprompt@gap>| !gapinput@tc := TwistedConjugation( phi, psi );;|
  !gapprompt@gap>| !gapinput@g1 := (4,6,5);;|
  !gapprompt@gap>| !gapinput@g2 := (1,6,4,2)(3,5);;|
  !gapprompt@gap>| !gapinput@IsTwistedConjugate( psi, phi, g1, g2 );|
  false
  !gapprompt@gap>| !gapinput@h := RepresentativeTwistedConjugation( phi, psi, g1, g2 );|
  (1,2)
  !gapprompt@gap>| !gapinput@tc( g1, h ) = g2;|
  true
\end{Verbatim}
 }

 
\section{\textcolor{Chapter }{Reidemeister Classes}}\label{Chapter_Twisted_Conjugacy_Section_Reidemeister_Classes}
\logpage{[ 2, 2, 0 ]}
\hyperdef{L}{X78BF33217A10D7B1}{}
{
  The equivalence classes of the equivalence relation $\sim_{\varphi,\psi}$ are called the \emph{Reidemeister classes of $(\varphi,\psi)$} or the \emph{$(\varphi,\psi)$\texttt{\symbol{45}}twisted conjugacy classes}. We denote the Reidemeister class of $g \in G$ by $[g]_{\varphi,\psi}$. The number of Reidemeister classes is called the Reidemeister number $R(\varphi,\psi)$ and is always a positive integer or infinity. 

\subsection{\textcolor{Chapter }{ReidemeisterClass}}
\logpage{[ 2, 2, 1 ]}\nobreak
\label{ReidemeisterClassGroup}
\hyperdef{L}{X876AF10978E10641}{}
{\noindent\textcolor{FuncColor}{$\triangleright$\enspace\texttt{ReidemeisterClass({\mdseries\slshape hom1[, hom2], g})\index{ReidemeisterClass@\texttt{ReidemeisterClass}}
\label{ReidemeisterClass}
}\hfill{\scriptsize (function)}}\\
\noindent\textcolor{FuncColor}{$\triangleright$\enspace\texttt{TwistedConjugacyClass({\mdseries\slshape hom1[, hom2], g})\index{TwistedConjugacyClass@\texttt{TwistedConjugacyClass}}
\label{TwistedConjugacyClass}
}\hfill{\scriptsize (function)}}\\


 If \mbox{\texttt{\mdseries\slshape hom1}} and \mbox{\texttt{\mdseries\slshape hom2}} are group homomorphisms from a group H to a group G, this method creates the
Reidemeister class of the pair (\mbox{\texttt{\mdseries\slshape hom1}}, \mbox{\texttt{\mdseries\slshape hom2}}) with representative \mbox{\texttt{\mdseries\slshape g}}. The following attributes and operations are available: 
\begin{itemize}
\item  \texttt{Representative}, which returns \mbox{\texttt{\mdseries\slshape g}}, 
\item  \texttt{GroupHomomorphismsOfReidemeisterClass}, which returns the list [ \mbox{\texttt{\mdseries\slshape hom1}}, \mbox{\texttt{\mdseries\slshape hom2}} ], 
\item  \texttt{ActingDomain}, which returns the group H, 
\item  \texttt{FunctionAction}, which returns the twisted conjugacy action on G, 
\item  \texttt{Random}, which returns a random element belonging to the Reidemeister class, 
\item  \texttt{\texttt{\symbol{92}}in}, which can be used to test if an element belongs to the Reidemeister class, 
\item  \texttt{List}, which lists all elements in the Reidemeister class if there are finitely
many, otherwise returns \texttt{fail}, 
\item  \texttt{Size}, which gives the number of elements in the Reidemeister class, 
\item  \texttt{StabiliserOfExternalSet}, which gives the stabiliser of the Reidemeister class under the twisted
conjugacy action. 
\end{itemize}
 

 }

 

\subsection{\textcolor{Chapter }{ReidemeisterClasses}}
\logpage{[ 2, 2, 2 ]}\nobreak
\label{ReidemeisterClassesGroup}
\hyperdef{L}{X823F37E384E924DD}{}
{\noindent\textcolor{FuncColor}{$\triangleright$\enspace\texttt{ReidemeisterClasses({\mdseries\slshape hom1[, hom2]})\index{ReidemeisterClasses@\texttt{ReidemeisterClasses}}
\label{ReidemeisterClasses}
}\hfill{\scriptsize (function)}}\\
\noindent\textcolor{FuncColor}{$\triangleright$\enspace\texttt{TwistedConjugacyClasses({\mdseries\slshape hom1[, hom2]})\index{TwistedConjugacyClasses@\texttt{TwistedConjugacyClasses}}
\label{TwistedConjugacyClasses}
}\hfill{\scriptsize (function)}}\\


 Returns a list containing the Reidemeister classes of ( \mbox{\texttt{\mdseries\slshape hom1}}, \mbox{\texttt{\mdseries\slshape hom2}} ) if the Reidemeister number $R( \mbox{\texttt{\mdseries\slshape hom1}}, \mbox{\texttt{\mdseries\slshape hom2}} )$ is finite, or returns \texttt{fail} otherwise. It is guaranteed that the Reidemeister class of the identity is in
the first position. 

 If $G$ is abelian, this function relies on (a generalisation of) \cite[Alg. 5]{dt21-a}. If $G$ and $H$ are finite and $G$ is not abelian, it relies on an orbit\texttt{\symbol{45}}stabiliser algorithm.
Otherwise, it relies on (variants of) \cite[Alg. 7]{dt21-a}. 

 This function is only guaranteed to produce a result if either $G = H$ or $G$ is nilpotent\texttt{\symbol{45}}by\texttt{\symbol{45}}finite. 

 }

 

\subsection{\textcolor{Chapter }{RepresentativesReidemeisterClasses}}
\logpage{[ 2, 2, 3 ]}\nobreak
\label{RepresentativesReidemeisterClassesGroup}
\hyperdef{L}{X818B93EC7BC2CBB3}{}
{\noindent\textcolor{FuncColor}{$\triangleright$\enspace\texttt{RepresentativesReidemeisterClasses({\mdseries\slshape hom1[, hom2]})\index{RepresentativesReidemeisterClasses@\texttt{RepresentativesReidemeisterClasses}}
\label{RepresentativesReidemeisterClasses}
}\hfill{\scriptsize (function)}}\\
\noindent\textcolor{FuncColor}{$\triangleright$\enspace\texttt{RepresentativesTwistedConjugacyClasses({\mdseries\slshape hom1[, hom2]})\index{RepresentativesTwistedConjugacyClasses@\texttt{Representatives}\-\texttt{Twisted}\-\texttt{Conjugacy}\-\texttt{Classes}}
\label{RepresentativesTwistedConjugacyClasses}
}\hfill{\scriptsize (function)}}\\


 Returns a list containing representatives of the Reidemeister classes of ( \mbox{\texttt{\mdseries\slshape hom1}}, \mbox{\texttt{\mdseries\slshape hom2}} ) if the Reidemeister number $R( \mbox{\texttt{\mdseries\slshape hom1}}, \mbox{\texttt{\mdseries\slshape hom2}} )$ is finite, or returns \texttt{fail} otherwise. It is guaranteed that the identity is in the first position. 

 The same remarks as for \texttt{ReidemeisterClasses} are valid here. 

 }

 

\subsection{\textcolor{Chapter }{ReidemeisterNumber}}
\logpage{[ 2, 2, 4 ]}\nobreak
\label{ReidemeisterNumberGroup}
\hyperdef{L}{X8330E244852075A7}{}
{\noindent\textcolor{FuncColor}{$\triangleright$\enspace\texttt{ReidemeisterNumber({\mdseries\slshape hom1[, hom2]})\index{ReidemeisterNumber@\texttt{ReidemeisterNumber}}
\label{ReidemeisterNumber}
}\hfill{\scriptsize (function)}}\\
\noindent\textcolor{FuncColor}{$\triangleright$\enspace\texttt{NrTwistedConjugacyClasses({\mdseries\slshape hom1[, hom2]})\index{NrTwistedConjugacyClasses@\texttt{NrTwistedConjugacyClasses}}
\label{NrTwistedConjugacyClasses}
}\hfill{\scriptsize (function)}}\\


 Returns the Reidemeister number of ( \mbox{\texttt{\mdseries\slshape hom1}}, \mbox{\texttt{\mdseries\slshape hom2}} ), i.e. the number of Reidemeister classes. 

 If $G$ is abelian, this function relies on (a generalisation of) \cite[Thm. 2.5]{jian83-a}. If $G = H$, $G$ is finite non\texttt{\symbol{45}}abelian and $\psi = \operatorname{id}_G$, it relies on \cite[Thm. 5]{fh94-a}. Otherwise, it uses the output of \texttt{ReidemeisterClasses}. 

 This function is only guaranteed to produce a result if either $G = H$ or $G$ is nilpotent\texttt{\symbol{45}}by\texttt{\symbol{45}}finite. 

 }

 
\begin{Verbatim}[commandchars=!@|,fontsize=\small,frame=single,label=Example]
  !gapprompt@gap>| !gapinput@tcc := ReidemeisterClass( phi, psi, g1 );|
  (4,6,5)^G
  !gapprompt@gap>| !gapinput@Representative( tcc );|
  (4,6,5)
  !gapprompt@gap>| !gapinput@GroupHomomorphismsOfReidemeisterClass( tcc );|
  [ [ (1,2)(3,5,4), (2,3)(4,5) ] -> [ (1,2)(3,4), () ],
  [ (1,2)(3,5,4), (2,3)(4,5) ] -> [ (1,4)(3,6), () ] ]
  !gapprompt@gap>| !gapinput@ActingDomain( tcc ) = H;|
  true
  !gapprompt@gap>| !gapinput@FunctionAction( tcc )( g1, h );|
  (1,6,4,2)(3,5)
  !gapprompt@gap>| !gapinput@Random( tcc ) in tcc;|
  true
  !gapprompt@gap>| !gapinput@List( tcc );|
  [ (4,6,5), (1,6,4,2)(3,5) ]
  !gapprompt@gap>| !gapinput@Size( tcc );|
  2
  !gapprompt@gap>| !gapinput@StabiliserOfExternalSet( tcc );|
  Group([ (1,2,3,4,5), (1,3,4,5,2) ])
  !gapprompt@gap>| !gapinput@ReidemeisterClasses( phi, psi ){[1..7]};|
  [ ()^G, (4,5,6)^G, (4,6,5)^G, (3,4)(5,6)^G, (3,4,5)^G, (3,4,6)^G, (3,5,4)^G ]
  !gapprompt@gap>| !gapinput@RepresentativesReidemeisterClasses( phi, psi ){[1..7]};|
  [ (), (4,5,6), (4,6,5), (3,4)(5,6), (3,4,5), (3,4,6), (3,5,4) ]
  !gapprompt@gap>| !gapinput@NrTwistedConjugacyClasses( phi, psi );|
  184
\end{Verbatim}
 }

 
\section{\textcolor{Chapter }{Reidemeister Spectra}}\label{Chapter_Twisted_Conjugacy_Section_Reidemeister_Spectra}
\logpage{[ 2, 3, 0 ]}
\hyperdef{L}{X7CED57E379712C3A}{}
{
  The set of all Reidemeister numbers of automorphisms is called the \emph{Reidemeister spectrum} and is denoted by $\operatorname{Spec}_R(G)$, i.e. 
\[\operatorname{Spec}_R(G) := \{\, R(\varphi) \mid \varphi \in
\operatorname{Aut}(G) \,\}.\]
 The set of all Reidemeister numbers of endomorphisms is called the \emph{extended Reidemeister spectrum} and is denoted by $\operatorname{ESpec}_R(G)$, i.e. 
\[\operatorname{ESpec}_R(G) := \{\, R(\varphi) \mid \varphi \in
\operatorname{End}(G) \,\}.\]
 The set of all Reidemeister numbers of pairs of homomorphisms from a group $H$ to a group $G$ is called the \emph{coincidence Reidemeister spectrum} of $H$ and $G$ and is denoted by $\operatorname{CSpec}_R(H,G)$, i.e. 
\[\operatorname{CSpec}_R(H,G) := \{\, R(\varphi, \psi) \mid \varphi,\psi \in
\operatorname{Hom}(H,G) \,\}.\]
 If \mbox{\texttt{\mdseries\slshape H}} = \mbox{\texttt{\mdseries\slshape G}} this is also denoted by $\operatorname{CSpec}_R(G)$. The set of all Reidemeister numbers of pairs of homomorphisms from every
group $H$ to a group $G$ is called the \emph{total Reidemeister spectrum} and is denoted by $\operatorname{TSpec}_R(G)$, i.e. 
\[\operatorname{TSpec}_R(G) := \bigcup_{H} \operatorname{CSpec}_R(H,G).\]
 

 Please note that the functions below are only implemented for finite groups. 

\subsection{\textcolor{Chapter }{ReidemeisterSpectrum}}
\logpage{[ 2, 3, 1 ]}\nobreak
\hyperdef{L}{X8777B3F77DBF01AF}{}
{\noindent\textcolor{FuncColor}{$\triangleright$\enspace\texttt{ReidemeisterSpectrum({\mdseries\slshape G})\index{ReidemeisterSpectrum@\texttt{ReidemeisterSpectrum}}
\label{ReidemeisterSpectrum}
}\hfill{\scriptsize (function)}}\\


 Returns the Reidemeister spectrum of \mbox{\texttt{\mdseries\slshape G}}. 

 If $G$ is abelian, this function relies on the results from \cite{send23-a}. }

 

\subsection{\textcolor{Chapter }{ExtendedReidemeisterSpectrum}}
\logpage{[ 2, 3, 2 ]}\nobreak
\hyperdef{L}{X8122B246860C1617}{}
{\noindent\textcolor{FuncColor}{$\triangleright$\enspace\texttt{ExtendedReidemeisterSpectrum({\mdseries\slshape G})\index{ExtendedReidemeisterSpectrum@\texttt{ExtendedReidemeisterSpectrum}}
\label{ExtendedReidemeisterSpectrum}
}\hfill{\scriptsize (function)}}\\


 Returns the extended Reidemeister spectrum of \mbox{\texttt{\mdseries\slshape G}}. }

 

\subsection{\textcolor{Chapter }{CoincidenceReidemeisterSpectrum}}
\logpage{[ 2, 3, 3 ]}\nobreak
\hyperdef{L}{X78839C0886EBDB71}{}
{\noindent\textcolor{FuncColor}{$\triangleright$\enspace\texttt{CoincidenceReidemeisterSpectrum({\mdseries\slshape [H, ]G})\index{CoincidenceReidemeisterSpectrum@\texttt{CoincidenceReidemeisterSpectrum}}
\label{CoincidenceReidemeisterSpectrum}
}\hfill{\scriptsize (function)}}\\


 Returns the coincidence Reidemeister spectrum of \mbox{\texttt{\mdseries\slshape H}} and \mbox{\texttt{\mdseries\slshape G}}. }

 

\subsection{\textcolor{Chapter }{TotalReidemeisterSpectrum}}
\logpage{[ 2, 3, 4 ]}\nobreak
\hyperdef{L}{X7DB417F182B155C5}{}
{\noindent\textcolor{FuncColor}{$\triangleright$\enspace\texttt{TotalReidemeisterSpectrum({\mdseries\slshape G})\index{TotalReidemeisterSpectrum@\texttt{TotalReidemeisterSpectrum}}
\label{TotalReidemeisterSpectrum}
}\hfill{\scriptsize (function)}}\\


 Returns the total Reidemeister spectrum of \mbox{\texttt{\mdseries\slshape G}}. }

 
\begin{Verbatim}[commandchars=!@|,fontsize=\small,frame=single,label=Example]
  !gapprompt@gap>| !gapinput@Q := QuaternionGroup( 8 );;|
  !gapprompt@gap>| !gapinput@D := DihedralGroup( 8 );;|
  !gapprompt@gap>| !gapinput@ReidemeisterSpectrum( Q );|
  [ 2, 3, 5 ]
  !gapprompt@gap>| !gapinput@ExtendedReidemeisterSpectrum( Q );|
  [ 1, 2, 3, 5 ]
  !gapprompt@gap>| !gapinput@CoincidenceReidemeisterSpectrum( Q );|
  [ 1, 2, 3, 4, 5, 8 ]
  !gapprompt@gap>| !gapinput@CoincidenceReidemeisterSpectrum( D, Q );|
  [ 4, 8 ]
  !gapprompt@gap>| !gapinput@CoincidenceReidemeisterSpectrum( Q, D );|
  [ 2, 3, 4, 6, 8 ]
  !gapprompt@gap>| !gapinput@TotalReidemeisterSpectrum( Q );|
  [ 1, 2, 3, 4, 5, 6, 8 ]
\end{Verbatim}
 }

 
\section{\textcolor{Chapter }{Reidemeister Zeta Functions}}\label{Chapter_Twisted_Conjugacy_Section_Reidemeister_Zeta_Functions}
\logpage{[ 2, 4, 0 ]}
\hyperdef{L}{X862C248A828A2C4A}{}
{
  Let $\varphi,\psi\colon G \to G$ be endomorphisms such that $R(\varphi^n,\psi^n) < \infty$ for all $n \in \mathbb{N}$. Then the \emph{Reidemeister zeta function} $Z_{\varphi,\psi}(s)$ of the pair $(\varphi,\psi)$ is defined as 
\[Z_{\varphi,\psi}(s) := \exp \sum_{n=1}^\infty \frac{R(\varphi^n,\psi^n)}{n}
s^n.\]
 

 Please note that the functions below are only implemented for endomorphisms of
finite groups. 

\subsection{\textcolor{Chapter }{ReidemeisterZetaCoefficients}}
\logpage{[ 2, 4, 1 ]}\nobreak
\label{ReidemeisterZetaCoefficientsGroup}
\hyperdef{L}{X78F0CE5987B70AA2}{}
{\noindent\textcolor{FuncColor}{$\triangleright$\enspace\texttt{ReidemeisterZetaCoefficients({\mdseries\slshape endo1[, endo2]})\index{ReidemeisterZetaCoefficients@\texttt{ReidemeisterZetaCoefficients}}
\label{ReidemeisterZetaCoefficients}
}\hfill{\scriptsize (function)}}\\


 For a finite group, the sequence of Reidemeister numbers of the iterates of \mbox{\texttt{\mdseries\slshape endo1}} and \mbox{\texttt{\mdseries\slshape endo2}}, i.e. the sequence $R(\mbox{\texttt{\mdseries\slshape endo1}},\mbox{\texttt{\mdseries\slshape endo2}})$, $R(\mbox{\texttt{\mdseries\slshape endo1}}^2,\mbox{\texttt{\mdseries\slshape endo2}}^2)$, ..., is eventually periodic, i.e. there exist a periodic sequence $(P_n)_{n \in \mathbb{N}}$ and an eventually zero sequence $(Q_n)_{n \in \mathbb{N}}$ such that 
\[\forall n \in \mathbb{N}: R(\varphi^n,\psi^n) = P_n + Q_n.\]
 This function returns a list containing two sublists: the first sublist
contains one period of the sequence $(P_n)_{n \in \mathbb{N}}$, the second sublist contains $(Q_n)_{n \in \mathbb{N}}$ up to the part where it becomes the constant zero sequence. }

 

\subsection{\textcolor{Chapter }{IsRationalReidemeisterZeta}}
\logpage{[ 2, 4, 2 ]}\nobreak
\label{IsRationalReidemeisterZetaGroup}
\hyperdef{L}{X79A2CD257BA1E037}{}
{\noindent\textcolor{FuncColor}{$\triangleright$\enspace\texttt{IsRationalReidemeisterZeta({\mdseries\slshape endo1[, endo2]})\index{IsRationalReidemeisterZeta@\texttt{IsRationalReidemeisterZeta}}
\label{IsRationalReidemeisterZeta}
}\hfill{\scriptsize (function)}}\\


 Returns \texttt{true} if the Reidemeister zeta function of \mbox{\texttt{\mdseries\slshape endo1}} and \mbox{\texttt{\mdseries\slshape endo2}} is rational, and \texttt{false} otherwise. }

 

\subsection{\textcolor{Chapter }{ReidemeisterZeta}}
\logpage{[ 2, 4, 3 ]}\nobreak
\label{ReidemeisterZetaGroup}
\hyperdef{L}{X7959DBAF78CC4401}{}
{\noindent\textcolor{FuncColor}{$\triangleright$\enspace\texttt{ReidemeisterZeta({\mdseries\slshape endo1[, endo2]})\index{ReidemeisterZeta@\texttt{ReidemeisterZeta}}
\label{ReidemeisterZeta}
}\hfill{\scriptsize (function)}}\\


 Returns the Reidemeister zeta function of \mbox{\texttt{\mdseries\slshape endo1}} and \mbox{\texttt{\mdseries\slshape endo2}} if it is rational, and \texttt{fail} otherwise. }

 

\subsection{\textcolor{Chapter }{PrintReidemeisterZeta}}
\logpage{[ 2, 4, 4 ]}\nobreak
\label{PrintReidemeisterZetaGroup}
\hyperdef{L}{X829058F97A8858F1}{}
{\noindent\textcolor{FuncColor}{$\triangleright$\enspace\texttt{PrintReidemeisterZeta({\mdseries\slshape endo1[, endo2]})\index{PrintReidemeisterZeta@\texttt{PrintReidemeisterZeta}}
\label{PrintReidemeisterZeta}
}\hfill{\scriptsize (function)}}\\


 Returns a string describing the Reidemeister zeta function of \mbox{\texttt{\mdseries\slshape endo1}} and \mbox{\texttt{\mdseries\slshape endo2}}. This is often more readable than evaluating \texttt{ReidemeisterZeta} in an indeterminate, and does not require rationality. }

 
\begin{Verbatim}[commandchars=!@|,fontsize=\small,frame=single,label=Example]
  !gapprompt@gap>| !gapinput@khi := GroupHomomorphismByImages( G, G, [ (1,2,3,4,5), (4,5,6) ],|
  !gapprompt@>| !gapinput@ [ (1,2,6,3,5), (1,4,5) ] );;|
  !gapprompt@gap>| !gapinput@ReidemeisterZetaCoefficients( khi );|
  [ [ 7 ], [  ] ]
  !gapprompt@gap>| !gapinput@IsRationalReidemeisterZeta( khi );|
  true
  !gapprompt@gap>| !gapinput@ReidemeisterZeta( khi );|
  function( s ) ... end
  !gapprompt@gap>| !gapinput@s := Indeterminate( Rationals, "s" );;|
  !gapprompt@gap>| !gapinput@ReidemeisterZeta( khi )(s);|
  (1)/(-s^7+7*s^6-21*s^5+35*s^4-35*s^3+21*s^2-7*s+1)
  !gapprompt@gap>| !gapinput@PrintReidemeisterZeta( khi );|
  "(1-s)^(-7)"
\end{Verbatim}
 }

 }

   
\chapter{\textcolor{Chapter }{Multiple Twisted Conjugacy Problem}}\label{Chapter_mult}
\logpage{[ 3, 0, 0 ]}
\hyperdef{L}{X8583B2B97FEA5615}{}
{
  
\section{\textcolor{Chapter }{The Multiple Twisted Conjugacy Problem}}\label{Chapter_Multiple_Twisted_Conjugacy_Problem_Section_The_Multiple_Twisted_Conjugacy_Problem}
\logpage{[ 3, 1, 0 ]}
\hyperdef{L}{X7B5C499E870B22B1}{}
{
  Let $H$ and $G_1, \ldots, G_n$ be groups. For each $i \in \{1,\ldots,n\}$, let $g_i,g_i' \in G_i$ and let $\varphi_i,\psi_i\colon H \to G_i$ be group homomorphisms. The multiple twisted conjugacy problem is the problem
of finding some $h \in H$ such that $g_i = \psi_i(h)g_i'\varphi_i(h)^{-1}$ for all $i \in \{1,\ldots,n\}$. 

\subsection{\textcolor{Chapter }{IsTwistedConjugateMultiple}}
\logpage{[ 3, 1, 1 ]}\nobreak
\hyperdef{L}{X81B53D0583708C51}{}
{\noindent\textcolor{FuncColor}{$\triangleright$\enspace\texttt{IsTwistedConjugateMultiple({\mdseries\slshape hom1List[, hom2List], g1List[, g2List]})\index{IsTwistedConjugateMultiple@\texttt{IsTwistedConjugateMultiple}}
\label{IsTwistedConjugateMultiple}
}\hfill{\scriptsize (function)}}\\


 Verifies whether the multiple twisted conjugacy problem for the given
homomorphisms and elements has a solution. }

 

\subsection{\textcolor{Chapter }{RepresentativeTwistedConjugationMultiple}}
\logpage{[ 3, 1, 2 ]}\nobreak
\hyperdef{L}{X83FD43F287FD5990}{}
{\noindent\textcolor{FuncColor}{$\triangleright$\enspace\texttt{RepresentativeTwistedConjugationMultiple({\mdseries\slshape hom1List[, hom2List], g1List[, g2List]})\index{RepresentativeTwistedConjugationMultiple@\texttt{Representative}\-\texttt{Twisted}\-\texttt{Conjugation}\-\texttt{Multiple}}
\label{RepresentativeTwistedConjugationMultiple}
}\hfill{\scriptsize (function)}}\\


 Computes a solution to the multiple twisted conjugacy problem for the given
homomorphisms and elements, or returns \texttt{fail} if no solution exists. }

 
\begin{Verbatim}[commandchars=!@|,fontsize=\small,frame=single,label=Example]
  !gapprompt@gap>| !gapinput@H := SymmetricGroup( 5 );;|
  !gapprompt@gap>| !gapinput@G := AlternatingGroup( 6 );;|
  !gapprompt@gap>| !gapinput@tau := GroupHomomorphismByImages( H, G, [ (1,2)(3,5,4), (2,3)(4,5) ],|
  !gapprompt@>| !gapinput@ [ (1,3)(4,6), () ] );;|
  !gapprompt@gap>| !gapinput@phi := GroupHomomorphismByImages( H, G, [ (1,2)(3,5,4), (2,3)(4,5) ],|
  !gapprompt@>| !gapinput@ [ (1,2)(3,6), () ] );;|
  !gapprompt@gap>| !gapinput@psi := GroupHomomorphismByImages( H, G, [ (1,2)(3,5,4), (2,3)(4,5) ],|
  !gapprompt@>| !gapinput@ [ (1,4)(3,6), () ] );;|
  !gapprompt@gap>| !gapinput@khi := GroupHomomorphismByImages( H, G, [ (1,2)(3,5,4), (2,3)(4,5) ],|
  !gapprompt@>| !gapinput@ [ (1,2)(3,4), () ] );;|
  !gapprompt@gap>| !gapinput@IsTwistedConjugateMultiple( [ tau, phi ], [ psi, khi ],|
  !gapprompt@>| !gapinput@ [ (1,5)(4,6), (1,4)(3,5) ], [ (1,4,5,3,6), (2,4,5,6,3) ] );|
  true
  !gapprompt@gap>| !gapinput@RepresentativeTwistedConjugationMultiple( [ tau, phi ], [ psi, khi ],|
  !gapprompt@>| !gapinput@ [ (1,5)(4,6), (1,4)(3,5) ], [ (1,4,5,3,6), (2,4,5,6,3) ] );|
  (1,2)
\end{Verbatim}
 }

 }

   
\chapter{\textcolor{Chapter }{Homomorphisms}}\label{Chapter_homs}
\logpage{[ 4, 0, 0 ]}
\hyperdef{L}{X84975388859F203D}{}
{
  
\section{\textcolor{Chapter }{Representatives of homomorphisms between groups}}\label{Chapter_Homomorphisms_Section_Representatives_of_homomorphisms_between_groups}
\logpage{[ 4, 1, 0 ]}
\hyperdef{L}{X80DDEC8C82E2A4F1}{}
{
  Please note that the functions below are only implemented for finite groups. 

\subsection{\textcolor{Chapter }{RepresentativesAutomorphismClasses}}
\logpage{[ 4, 1, 1 ]}\nobreak
\hyperdef{L}{X78ADEE0C83819159}{}
{\noindent\textcolor{FuncColor}{$\triangleright$\enspace\texttt{RepresentativesAutomorphismClasses({\mdseries\slshape G})\index{RepresentativesAutomorphismClasses@\texttt{RepresentativesAutomorphismClasses}}
\label{RepresentativesAutomorphismClasses}
}\hfill{\scriptsize (function)}}\\


 Let \mbox{\texttt{\mdseries\slshape G}} be a group. This command returns a list of the automorphisms of \mbox{\texttt{\mdseries\slshape G}} up to composition with inner automorphisms. }

 

\subsection{\textcolor{Chapter }{RepresentativesEndomorphismClasses}}
\logpage{[ 4, 1, 2 ]}\nobreak
\hyperdef{L}{X7A7935B286050886}{}
{\noindent\textcolor{FuncColor}{$\triangleright$\enspace\texttt{RepresentativesEndomorphismClasses({\mdseries\slshape G})\index{RepresentativesEndomorphismClasses@\texttt{RepresentativesEndomorphismClasses}}
\label{RepresentativesEndomorphismClasses}
}\hfill{\scriptsize (function)}}\\


 Let \mbox{\texttt{\mdseries\slshape G}} be a group. This command returns a list of the endomorphisms of \mbox{\texttt{\mdseries\slshape G}} up to composition with inner automorphisms. This does the same as calling \texttt{AllHomomorphismClasses(\mbox{\texttt{\mdseries\slshape G}},\mbox{\texttt{\mdseries\slshape G}})}, but should be faster for abelian and
non\texttt{\symbol{45}}2\texttt{\symbol{45}}generated groups. For
2\texttt{\symbol{45}}generated groups, this function takes its source code
from \texttt{AllHomomorphismClasses}. }

 

\subsection{\textcolor{Chapter }{RepresentativesHomomorphismClasses}}
\logpage{[ 4, 1, 3 ]}\nobreak
\hyperdef{L}{X81E5CF92816BF199}{}
{\noindent\textcolor{FuncColor}{$\triangleright$\enspace\texttt{RepresentativesHomomorphismClasses({\mdseries\slshape H, G})\index{RepresentativesHomomorphismClasses@\texttt{RepresentativesHomomorphismClasses}}
\label{RepresentativesHomomorphismClasses}
}\hfill{\scriptsize (function)}}\\


 Let \mbox{\texttt{\mdseries\slshape G}} and \mbox{\texttt{\mdseries\slshape H}} be groups. This command returns a list of the homomorphisms from \mbox{\texttt{\mdseries\slshape H}} to \mbox{\texttt{\mdseries\slshape G}}, up to composition with inner automorphisms of \mbox{\texttt{\mdseries\slshape G}}. This does the same as calling \texttt{AllHomomorphismClasses(\mbox{\texttt{\mdseries\slshape H}},\mbox{\texttt{\mdseries\slshape G}})}, but should be faster for abelian and
non\texttt{\symbol{45}}2\texttt{\symbol{45}}generated groups. For
2\texttt{\symbol{45}}generated groups, this function takes its source code
from \texttt{AllHomomorphismClasses}. }

 
\begin{Verbatim}[commandchars=!@|,fontsize=\small,frame=single,label=Example]
  !gapprompt@gap>| !gapinput@G := SymmetricGroup( 6 );;|
  !gapprompt@gap>| !gapinput@Auts := RepresentativesAutomorphismClasses( G );;|
  !gapprompt@gap>| !gapinput@Size( Auts );|
  2
  !gapprompt@gap>| !gapinput@ForAll( Auts, IsGroupHomomorphism and IsEndoMapping and IsBijective );|
  true
  !gapprompt@gap>| !gapinput@Ends := RepresentativesEndomorphismClasses( G );;|
  !gapprompt@gap>| !gapinput@Size( Ends );|
  6
  !gapprompt@gap>| !gapinput@ForAll( Ends, IsGroupHomomorphism and IsEndoMapping );|
  true
  !gapprompt@gap>| !gapinput@H := SymmetricGroup( 5 );;|
  !gapprompt@gap>| !gapinput@Homs := RepresentativesHomomorphismClasses( H, G );;|
  !gapprompt@gap>| !gapinput@Size( Homs );|
  6
  !gapprompt@gap>| !gapinput@ForAll( Homs, IsGroupHomomorphism );|
  true
\end{Verbatim}
 }

 
\section{\textcolor{Chapter }{Coincidence and Fixed Point Groups}}\label{Chapter_Homomorphisms_Section_Coincidence_and_Fixed_Point_Groups}
\logpage{[ 4, 2, 0 ]}
\hyperdef{L}{X8164A34A86155DFB}{}
{
  

\subsection{\textcolor{Chapter }{FixedPointGroup}}
\logpage{[ 4, 2, 1 ]}\nobreak
\hyperdef{L}{X799546928394FF8B}{}
{\noindent\textcolor{FuncColor}{$\triangleright$\enspace\texttt{FixedPointGroup({\mdseries\slshape endo})\index{FixedPointGroup@\texttt{FixedPointGroup}}
\label{FixedPointGroup}
}\hfill{\scriptsize (function)}}\\


 Let \mbox{\texttt{\mdseries\slshape endo}} be an endomorphism of a group G. This command returns the subgroup of G
consisting of the elements fixed under the endomorphism \mbox{\texttt{\mdseries\slshape endo}}. 

 This function does the same as \texttt{CoincidenceGroup}(\mbox{\texttt{\mdseries\slshape endo}},$\operatorname{id}_G$). }

 

\subsection{\textcolor{Chapter }{CoincidenceGroup}}
\logpage{[ 4, 2, 2 ]}\nobreak
\hyperdef{L}{X780DF6247E3E9190}{}
{\noindent\textcolor{FuncColor}{$\triangleright$\enspace\texttt{CoincidenceGroup({\mdseries\slshape hom1, hom2[, ...]})\index{CoincidenceGroup@\texttt{CoincidenceGroup}}
\label{CoincidenceGroup}
}\hfill{\scriptsize (function)}}\\


 Let \mbox{\texttt{\mdseries\slshape hom1}}, \mbox{\texttt{\mdseries\slshape hom2}}, ... be group homomorphisms from a group H to a group G. This command returns
the subgroup of H consisting of the elements h for which h\texttt{\symbol{94}}\mbox{\texttt{\mdseries\slshape hom1}} = h\texttt{\symbol{94}}\mbox{\texttt{\mdseries\slshape hom2}} = ... 

 For infinite non\texttt{\symbol{45}}abelian groups, this function relies on a
mixture of the algorithms described in \cite[Thm. 2]{roma16-a}, \cite[Sec. 5.4]{bkl20-a} and \cite[Sec. 7]{roma21-a}. }

 
\begin{Verbatim}[commandchars=!@|,fontsize=\small,frame=single,label=Example]
  !gapprompt@gap>| !gapinput@phi := GroupHomomorphismByImages( G, G, [ (1,2,5,6,4), (1,2)(3,6)(4,5) ],|
  !gapprompt@>| !gapinput@ [ (2,3,4,5,6), (1,2) ] );;|
  !gapprompt@gap>| !gapinput@Set( FixedPointGroup( phi ) );|
  [ (), (1,2,3,6,5), (1,3,5,2,6), (1,5,6,3,2), (1,6,2,5,3) ]
  !gapprompt@gap>| !gapinput@psi := GroupHomomorphismByImages( H, G, [ (1,2,3,4,5), (1,2) ],|
  !gapprompt@>| !gapinput@ [ (), (1,2) ] );;|
  !gapprompt@gap>| !gapinput@khi := GroupHomomorphismByImages( H, G, [ (1,2,3,4,5), (1,2) ],|
  !gapprompt@>| !gapinput@ [ (), (1,2)(3,4) ] );;|
  !gapprompt@gap>| !gapinput@CoincidenceGroup( psi, khi ) = AlternatingGroup( 5 );|
  true
\end{Verbatim}
 }

 
\section{\textcolor{Chapter }{Induced and restricted group homomorphisms}}\label{Chapter_Homomorphisms_Section_Induced_and_restricted_group_homomorphisms}
\logpage{[ 4, 3, 0 ]}
\hyperdef{L}{X8084A06782AE362E}{}
{
  

\subsection{\textcolor{Chapter }{InducedHomomorphism}}
\logpage{[ 4, 3, 1 ]}\nobreak
\hyperdef{L}{X7F6D0625837B7B94}{}
{\noindent\textcolor{FuncColor}{$\triangleright$\enspace\texttt{InducedHomomorphism({\mdseries\slshape epi1, epi2, hom})\index{InducedHomomorphism@\texttt{InducedHomomorphism}}
\label{InducedHomomorphism}
}\hfill{\scriptsize (function)}}\\


 Let \mbox{\texttt{\mdseries\slshape hom}} be a group homomorphism from a group H to a group G, let \mbox{\texttt{\mdseries\slshape epi1}} be an epimorphism from H to a group Q and let \mbox{\texttt{\mdseries\slshape epi2}} be an epimorphism from G to a group P such that the kernel of \mbox{\texttt{\mdseries\slshape epi1}} is mapped into the kernel of \mbox{\texttt{\mdseries\slshape epi2}} by \mbox{\texttt{\mdseries\slshape hom}}. This command returns the homomorphism from Q to P induced by \mbox{\texttt{\mdseries\slshape hom}} via \mbox{\texttt{\mdseries\slshape epi1}} and \mbox{\texttt{\mdseries\slshape epi2}}, that is, the homomorphism from Q to P which maps h\texttt{\texttt{\symbol{94}}\mbox{\texttt{\mdseries\slshape epi1}}} to \texttt{(}h\texttt{\texttt{\symbol{94}}\mbox{\texttt{\mdseries\slshape hom}})\texttt{\symbol{94}}\mbox{\texttt{\mdseries\slshape epi2}}}, for any element h of H. This generalises \texttt{InducedAutomorphism} to homomorphisms. }

 

\subsection{\textcolor{Chapter }{RestrictedHomomorphism}}
\logpage{[ 4, 3, 2 ]}\nobreak
\hyperdef{L}{X7DBA352982923900}{}
{\noindent\textcolor{FuncColor}{$\triangleright$\enspace\texttt{RestrictedHomomorphism({\mdseries\slshape hom, N, M})\index{RestrictedHomomorphism@\texttt{RestrictedHomomorphism}}
\label{RestrictedHomomorphism}
}\hfill{\scriptsize (function)}}\\


 Let \mbox{\texttt{\mdseries\slshape hom}} be a group homomorphism from a group H to a group G, and let \mbox{\texttt{\mdseries\slshape N}} be subgroup of H such that its image under \mbox{\texttt{\mdseries\slshape hom}} is a subgroup of \mbox{\texttt{\mdseries\slshape M}}. This command returns the homomorphism from N to M induced by \mbox{\texttt{\mdseries\slshape hom}}. This is similar to \texttt{RestrictedMapping}, but the range is explicitly set to \mbox{\texttt{\mdseries\slshape M}}. }

 
\begin{Verbatim}[commandchars=!@|,fontsize=\small,frame=single,label=Example]
  !gapprompt@gap>| !gapinput@G := PcGroupCode( 1018013, 28 );;|
  !gapprompt@gap>| !gapinput@phi := GroupHomomorphismByImages( G, G, [ G.1, G.3 ],|
  !gapprompt@>| !gapinput@ [ G.1*G.2*G.3^2, G.3^4 ] );;|
  !gapprompt@gap>| !gapinput@N := DerivedSubgroup( G );;|
  !gapprompt@gap>| !gapinput@p := NaturalHomomorphismByNormalSubgroup( G, N );|
  [ f1, f2, f3 ] -> [ f1, f2, <identity> of ... ]
  !gapprompt@gap>| !gapinput@ind := InducedHomomorphism( p, p, phi );|
  [ f1 ] -> [ f1*f2 ]
  !gapprompt@gap>| !gapinput@Source( ind ) = Range( p ) and Range( ind ) = Range( p );|
  true
  !gapprompt@gap>| !gapinput@res := RestrictedHomomorphism( phi, N, N );|
  [ f3 ] -> [ f3^4 ]
  !gapprompt@gap>| !gapinput@Source( res ) = N and Range( res ) = N;|
  true
\end{Verbatim}
 }

 }

   
\chapter{\textcolor{Chapter }{Cosets}}\label{Chapter_csts}
\logpage{[ 5, 0, 0 ]}
\hyperdef{L}{X81002AA87DDBC02F}{}
{
  Please note that the functions below are implemented only for PcpGroups. 
\section{\textcolor{Chapter }{Intersection of cosets in PcpGroups}}\label{Chapter_Cosets_Section_Intersection_of_cosets_in_PcpGroups}
\logpage{[ 5, 1, 0 ]}
\hyperdef{L}{X7B633416846B090A}{}
{
  

\subsection{\textcolor{Chapter }{Intersection}}
\logpage{[ 5, 1, 1 ]}\nobreak
\label{IntersectionCosets}
\hyperdef{L}{X851069107CACF98E}{}
{\noindent\textcolor{FuncColor}{$\triangleright$\enspace\texttt{Intersection({\mdseries\slshape C1, C2, ...})\index{Intersection@\texttt{Intersection}}
\label{Intersection}
}\hfill{\scriptsize (function)}}\\
\noindent\textcolor{FuncColor}{$\triangleright$\enspace\texttt{Intersection({\mdseries\slshape list})\index{Intersection@\texttt{Intersection}!for IsList}
\label{Intersection:for IsList}
}\hfill{\scriptsize (function)}}\\
\noindent\textcolor{FuncColor}{$\triangleright$\enspace\texttt{Intersection2({\mdseries\slshape C1, C2})\index{Intersection2@\texttt{Intersection2}!for IsRightCoset, IsRightCoset}
\label{Intersection2:for IsRightCoset, IsRightCoset}
}\hfill{\scriptsize (operation)}}\\


 Here, \mbox{\texttt{\mdseries\slshape C1}}, \mbox{\texttt{\mdseries\slshape C2}}, ... must be (right) cosets, or \mbox{\texttt{\mdseries\slshape list}} must be a list of (right) cosets. 

 }

 
\begin{Verbatim}[commandchars=!@|,fontsize=\small,frame=single,label=Example]
  !gapprompt@gap>| !gapinput@G := ExamplesOfSomePcpGroups( 5 );;|
  !gapprompt@gap>| !gapinput@H := Subgroup( G, [ G.1*G.2^-1*G.3^-1*G.4^-1, G.2^-1*G.3*G.4^-2 ] );;|
  !gapprompt@gap>| !gapinput@K := Subgroup( G, [ G.1*G.3^-2*G.4^2, G.1*G.4^4 ] );;|
  !gapprompt@gap>| !gapinput@x := G.1*G.3^-1;;|
  !gapprompt@gap>| !gapinput@y := G.1*G.2^-1*G.3^-2*G.4^-1;;|
  !gapprompt@gap>| !gapinput@Hx := RightCoset( H, x );;|
  !gapprompt@gap>| !gapinput@Ky := RightCoset( K, y );;|
  !gapprompt@gap>| !gapinput@Intersection( Hx, Ky );|
  RightCoset(<group with 2 generators>,<object>)
\end{Verbatim}
 }

 
\section{\textcolor{Chapter }{Membership in double cosets in PcpGroups}}\label{Chapter_Cosets_Section_Membership_in_double_cosets_in_PcpGroups}
\logpage{[ 5, 2, 0 ]}
\hyperdef{L}{X82368604816F26DC}{}
{
  

\subsection{\textcolor{Chapter }{\texttt{\symbol{92}}in (for IsPcpElement, IsDoubleCoset)}}
\logpage{[ 5, 2, 1 ]}\nobreak
\hyperdef{L}{X8297E4417E2192D3}{}
{\noindent\textcolor{FuncColor}{$\triangleright$\enspace\texttt{\texttt{\symbol{92}}in({\mdseries\slshape g, D})\index{\texttt{\symbol{92}}in@\texttt{\texttt{\symbol{92}}in}!for IsPcpElement, IsDoubleCoset}
\label{bSlashin:for IsPcpElement, IsDoubleCoset}
}\hfill{\scriptsize (operation)}}\\


 Here \mbox{\texttt{\mdseries\slshape g}} is an element of a polycyclic group and \mbox{\texttt{\mdseries\slshape D}} is a double coset in the same group. }

 
\begin{Verbatim}[commandchars=!@|,fontsize=\small,frame=single,label=Example]
  !gapprompt@gap>| !gapinput@HxK := DoubleCoset( H, x, K );;|
  !gapprompt@gap>| !gapinput@G.1 in HxK;|
  false
  !gapprompt@gap>| !gapinput@G.2 in HxK;|
  true
\end{Verbatim}
 }

 }

 \def\bibname{References\logpage{[ "Bib", 0, 0 ]}
\hyperdef{L}{X7A6F98FD85F02BFE}{}
}

\bibliographystyle{alpha}
\bibliography{manual.bib}

\addcontentsline{toc}{chapter}{References}

\def\indexname{Index\logpage{[ "Ind", 0, 0 ]}
\hyperdef{L}{X83A0356F839C696F}{}
}

\cleardoublepage
\phantomsection
\addcontentsline{toc}{chapter}{Index}


\printindex

\immediate\write\pagenrlog{["Ind", 0, 0], \arabic{page},}
\newpage
\immediate\write\pagenrlog{["End"], \arabic{page}];}
\immediate\closeout\pagenrlog
\end{document}
